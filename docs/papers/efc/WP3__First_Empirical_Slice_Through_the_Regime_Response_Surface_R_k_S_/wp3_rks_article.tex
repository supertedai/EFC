\documentclass[11pt,a4paper]{article}

% Packages
\usepackage[utf8]{inputenc}
\usepackage[T1]{fontenc}
\usepackage{amsmath,amssymb,amsfonts}
\usepackage{graphicx}
\usepackage{booktabs}
\usepackage{hyperref}
\usepackage{xcolor}
\usepackage[margin=2.5cm]{geometry}
\usepackage{natbib}
\usepackage{doi}

% Custom commands
\newcommand{\fsig}{f\sigma_8}
\newcommand{\Rks}{R(k,S)}
\newcommand{\LCDM}{$\Lambda$CDM}

% Hyperref setup
\hypersetup{
    colorlinks=true,
    linkcolor=blue,
    citecolor=blue,
    urlcolor=blue
}

\title{WP3: First Empirical Slice Through the Regime Response Surface $R(k,S)$}

\author{Morten Magnusson\\
\small \href{https://orcid.org/0009-0002-4860-5095}{ORCID: 0009-0002-4860-5095}\\
\small Independent Researcher
}

\date{January 30, 2026}

\begin{document}

\maketitle

\begin{abstract}
We present the first empirical mapping of the regime response surface $R(k,S)$ along the redshift-space distortion (RSD) trajectory. Using $f\sigma_8(z)$ measurements from BOSS DR12, eBOSS DR16, and DESI Y1 ($N=12$), we constrain one coordinate on the gravitational response surface: $R(k\approx 0.13\,h/\text{Mpc}, S\approx 0.30) \approx +0.30$. Under Akaike Information Criterion (AIC), \LCDM{} ($R=0$) is preferred by $\Delta\text{AIC} = -3.8$, indicating that the data are consistent with zero response within complexity penalty. This work is \emph{cartography}---a measurement in a new coordinate system---not a model comparison or validation claim. WP3 provides the first entry in a planned response atlas mapping gravitational behavior across structural regimes.
\end{abstract}

\noindent\textbf{Keywords:} Energy-Flow Cosmology, modified gravity, regime response surface, growth rate, $f\sigma_8$, RSD

\noindent\textbf{DOI:} \href{https://doi.org/10.6084/m9.figshare.31215259}{10.6084/m9.figshare.31215259}

\section{Introduction}

Standard approaches to testing modified gravity theories typically parameterize deviations from General Relativity using time-dependent functions such as $\mu(a)$ or scale-and-time-dependent functions $\mu(k,z)$. These parameterizations treat modifications as perturbations to be constrained, leading naturally to model comparison frameworks where modified gravity ``competes'' with \LCDM.

The $R(k,S)$ framework \citep{Magnusson2026RkS} introduces a fundamentally different perspective. Rather than parameterizing deviations from GR, it establishes a \emph{coordinate system} for gravitational response based on:
\begin{itemize}
    \item $k$: spatial scale (wavenumber)
    \item $S$: structural maturity (a state variable encoding entropy/structure evolution)
\end{itemize}

The gravitational response is then:
\begin{equation}
    \mu = 1 + R(k,S)
\end{equation}
where $R(k,S)$ is a response surface that can take positive, negative, or zero values depending on the structural regime.

This reframing transforms the question from ``Does modified gravity fit better than \LCDM?'' to ``Where does this observation sit in regime-response space?'' The result is \emph{cartography}, not competition.

\section{The $R(k,S)$ Framework}

\subsection{From Time to Structural Maturity}

In standard cosmology, evolution is parameterized by cosmic time or equivalently scale factor $a$ and redshift $z$. The $R(k,S)$ framework replaces this with \emph{structural maturity} $S$, defined as a state variable that tracks the universe's progression from singularity ($S=0$) to altularity ($S=1$).

For this WP3 feasibility slice, we adopt a monotonic proxy mapping:
\begin{equation}
    S(z) = \frac{1}{1 + (z/z_{\text{mid}})^2}
\end{equation}
where $z_{\text{mid}} \approx 1.5$ corresponds to peak structure formation.

\textbf{Important caveat:} This $S(z)$ is a convenient monotonic proxy for structural maturity, not yet a physically derived observable. In the full $R(k,S)$ framework, $S$ should be computed from nonlinear growth statistics (e.g., $\sigma/\sigma_{\text{lin}}$, halo mass function evolution, or entropy production rates). The present work demonstrates the methodology using this proxy; future work will replace it with observationally grounded $S$ definitions. Without this replacement, $S$ remains a reparameterized redshift.

\subsection{Scale Dependence}

Different observational probes are sensitive to different spatial scales $k$. Redshift-space distortions (RSD) primarily probe the quasi-linear regime at $k \sim 0.1\,h/\text{Mpc}$. The effective wavenumber varies mildly with redshift:
\begin{equation}
    k_{\text{eff}}(z) \approx 0.1 \times (1 + 0.15z) \quad [h/\text{Mpc}]
\end{equation}

\subsection{WP3 as a Trajectory}

WP3 does not test the entire $R(k,S)$ surface. It tests \emph{one trajectory} through this space:
\begin{equation}
    \text{WP3 trajectory:} \quad (k_{\text{eff}}(z), S(z)) \quad \text{for } z \in [0.3, 1.5]
\end{equation}

This corresponds to:
\begin{itemize}
    \item $k \in [0.1, 0.13]\,h/\text{Mpc}$
    \item $S \in [0.3, 0.7]$
\end{itemize}

\section{Data and Method}

\subsection{Dataset}

We use 12 $f\sigma_8(z)$ measurements from three surveys:

\begin{table}[h]
\centering
\caption{$f\sigma_8(z)$ data used in this analysis}
\label{tab:data}
\begin{tabular}{@{}lcccc@{}}
\toprule
Survey & $z$ & $f\sigma_8$ & $\sigma$ & Reference \\
\midrule
BOSS DR12 & 0.38 & 0.497 & 0.045 & Alam et al. (2017) \\
BOSS DR12 & 0.51 & 0.458 & 0.038 & Alam et al. (2017) \\
BOSS DR12 & 0.61 & 0.436 & 0.034 & Alam et al. (2017) \\
eBOSS DR16 & 0.70 & 0.473 & 0.041 & eBOSS (2021) \\
eBOSS DR16 & 0.85 & 0.315 & 0.095 & eBOSS (2021) \\
eBOSS DR16 & 1.48 & 0.462 & 0.045 & eBOSS (2021) \\
DESI Y1 & 0.295 & 0.447 & 0.028 & DESI (2024) \\
DESI Y1 & 0.510 & 0.470 & 0.025 & DESI (2024) \\
DESI Y1 & 0.706 & 0.447 & 0.022 & DESI (2024) \\
DESI Y1 & 0.930 & 0.437 & 0.024 & DESI (2024) \\
DESI Y1 & 1.184 & 0.389 & 0.032 & DESI (2024) \\
DESI Y1 & 1.491 & 0.297 & 0.055 & DESI (2024) \\
\bottomrule
\end{tabular}
\end{table}

\textbf{Limitation:} We use diagonal errors only; covariance matrices are not included. Data hash: \texttt{c76720490b99dbd6}.

\textbf{Statistical caveat:} Because no covariance matrices are used, the $\chi^2$, AIC, and BIC values reported here are approximate and should be interpreted as indicative only. A full analysis would require the published covariance matrices from each survey.

\subsection{Models}

\paragraph{\LCDM{} ($R=0$):}
\begin{equation}
    f\sigma_8(z) = \Omega_m(z)^\gamma \times \sigma_8 \times D(z)
\end{equation}
with 3 free parameters: $\sigma_8$, $\Omega_m$, $\gamma$.

\paragraph{$R(k,S)$ slice:}
\begin{equation}
    f\sigma_8(z) = \Omega_m(z)^\gamma \times \sqrt{\mu(z)} \times \sigma_8 \times D(z)
\end{equation}
where $\mu(z)$ is parameterized using a local Gaussian basis function to probe response amplitude along the trajectory:
\begin{equation}
    \mu(z) = 1 + \Delta R \exp\left[-\frac{(z - z_{\text{peak}})^2}{2 \times 0.6^2}\right]
\end{equation}
\textbf{Important caveat:} This Gaussian profile is a phenomenological basis function chosen for mathematical convenience, not an EFC prediction. It allows us to measure whether the data prefer non-zero response amplitude ($\Delta R \neq 0$) at some characteristic redshift ($z_{\text{peak}}$). The true $R(k,S)$ surface shape must ultimately be derived from first principles or reconstructed non-parametrically.

This model has 5 free parameters: $\sigma_8$, $\Omega_m$, $\Delta R$, $z_{\text{peak}}$, $\gamma$.

\subsection{Optimization}

We use differential evolution (seed=42, maxiter=1000, tol=$10^{-8}$) to minimize $\chi^2$.

\section{Results}

\begin{table}[h]
\centering
\caption{Model comparison results}
\label{tab:results}
\begin{tabular}{@{}lccccc@{}}
\toprule
Model & $k$ (params) & $\chi^2$ & dof & AIC & BIC \\
\midrule
\LCDM{} & 3 & 9.36 & 9 & 15.4 & 16.8 \\
$R(k,S)$ slice & 5 & 9.12 & 7 & 19.1 & 21.4 \\
\bottomrule
\end{tabular}
\end{table}

\paragraph{Best-fit parameters:}
\begin{itemize}
    \item \LCDM: $\sigma_8 = 0.775$, $\Omega_m = 0.250$, $\gamma = 0.450$
    \item $R(k,S)$: $\sigma_8 = 0.797$, $\Omega_m = 0.276$, $\Delta R = +0.30$, $z_{\text{peak}} = 2.30$, $\gamma = 0.61$
\end{itemize}

\paragraph{Model comparison:}
\begin{itemize}
    \item $\Delta\chi^2 = +0.24$ (marginal improvement for $R(k,S)$)
    \item $\Delta\text{AIC} = -3.8$ (\LCDM{} preferred)
    \item $\Delta\text{BIC} = -4.6$ (\LCDM{} preferred)
\end{itemize}

\paragraph{Coordinate on response surface:}
\begin{equation}
    R(k \approx 0.13\,h/\text{Mpc}, S \approx 0.30) \approx +0.30
\end{equation}

\section{Interpretation}

\subsection{What WP3 Shows}

WP3 demonstrates that:
\begin{enumerate}
    \item A non-zero response $R \approx +0.30$ is \emph{allowed} by the data at this trajectory
    \item However, $R = 0$ (\LCDM) is \emph{preferred} when complexity is penalized
    \item The measurement is \emph{prior-sensitive}: the location on the surface is not tightly constrained
\end{enumerate}

\subsection{What WP3 Does Not Show}

WP3 does \emph{not}:
\begin{itemize}
    \item Validate the $R(k,S)$ framework
    \item Falsify the $R(k,S)$ framework
    \item Claim that EFC is better than \LCDM
    \item Provide a complete map of the response surface
\end{itemize}

\subsection{Epistemic Status}

This is \textbf{cartography}: a measurement in a new coordinate system. The observable ($f\sigma_8$) becomes a coordinate; the theory becomes a geometry.

WP3 provides \emph{one noisy point} on the response surface---exactly what early cartography looks like.

\section{The Response Atlas}

WP3 is the first entry in a planned response atlas:

\begin{table}[h]
\centering
\caption{Response Atlas: planned trajectories through $R(k,S)$}
\label{tab:atlas}
\begin{tabular}{@{}llll@{}}
\toprule
Probe & $k$ window & $S$ range & Status \\
\midrule
RSD / $f\sigma_8$ & $\sim 0.1\,h/\text{Mpc}$ & $[0.3, 0.7]$ & \textbf{WP3 (this work)} \\
Weak lensing & different $k$ & overlapping $S$ & WP4 (ready) \\
Clusters & smaller $k$ & higher $S$ & Planned \\
CMB & large scales & $S \approx 0$ & Boundary condition \\
\bottomrule
\end{tabular}
\end{table}

Each probe maps a different trajectory through $(k,S)$ space. Together, they will constrain the shape of the response surface.

\section{Conclusion}

WP3 provides the first empirical coordinate on the regime response surface $R(k,S)$:
\begin{equation}
    R(k \approx 0.13, S \approx 0.30) \approx +0.30 \quad \text{(prior-sensitive, AIC prefers } R=0\text{)}
\end{equation}

This is not a validation or falsification of Energy-Flow Cosmology. It is a measurement that locates one class of observations within a new coordinate system for gravitational response.

The shift from ``Does modified gravity fit better?'' to ``Where does this observation sit in regime-response space?'' represents a move from curve fitting to state-space mapping---from parameter tuning to theory-driven cartography.

\section*{Data Availability}

Manifest and analysis code: \texttt{wp3\_rks\_manifest.json}\\
Data hash: \texttt{c76720490b99dbd6}\\
Run ID: \texttt{wp3\_rks\_v2.0}

\section*{Acknowledgments}

This work made use of public data from BOSS, eBOSS, and DESI collaborations.

\begin{thebibliography}{9}

\bibitem[Alam et al.(2017)]{Alam2017}
Alam, S., et al. 2017, MNRAS, 470, 2617

\bibitem[eBOSS Collaboration(2021)]{eBOSS2021}
eBOSS Collaboration 2021, PRD, 103, 083533

\bibitem[DESI Collaboration(2024)]{DESI2024}
DESI Collaboration 2024, arXiv:2404.03002

\bibitem[Magnusson(2026a)]{Magnusson2026RkS}
Magnusson, M. 2026a, ``$R(k,S)$ as a Regime Response Surface in Energy-Flow Cosmology,'' Figshare, \href{https://doi.org/10.6084/m9.figshare.31211437}{doi:10.6084/m9.figshare.31211437}

\end{thebibliography}

\end{document}
