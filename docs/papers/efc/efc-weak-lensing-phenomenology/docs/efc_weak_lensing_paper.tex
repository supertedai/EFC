\documentclass[11pt,a4paper]{article}
\usepackage[utf8]{inputenc}
\usepackage[T1]{fontenc}
\usepackage{amsmath,amssymb,amsfonts}
\usepackage{graphicx}
\usepackage{hyperref}
\usepackage{natbib}
\usepackage{geometry}
\geometry{margin=2.5cm}

\title{EFC Weak Lensing Phenomenology:\\
A Testable $\mu(k,z)$ Framework and DES Y6 Validation Protocol}

\author{Morten Magnusson\\
\small Independent Researcher, Norway\\
\small ORCID: 0009-0002-4860-5095\\
\small \href{https://energyflow-cosmology.com}{energyflow-cosmology.com}
}

\date{January 2026}

\begin{document}

\maketitle

\begin{abstract}
We present a phenomenological framework for testing Energy-Flow Cosmology (EFC) 
against weak gravitational lensing observations. Starting from the EFC field 
equations, we demonstrate that the published EFC field equations do not predict 
modified lensing without additional structure. We introduce a minimal operational 
closure (Postulat A) that linearly couples entropy production to matter 
density, yielding an explicit modified gravity parameter $\mu(k,z)$ in 
standard cosmological perturbation form. We derive the resulting weak 
lensing predictions and present a complete validation protocol against 
Dark Energy Survey Year 6 (DES Y6) 3$\times$2pt data, including explicit 
pass/fail criteria. This work does not claim that EFC explains the $S_8$ 
tension; rather, it provides the minimal theoretical infrastructure required 
to test whether it can. All code is publicly available.
\end{abstract}

\section{Introduction}

Energy-Flow Cosmology (EFC) is a field-theoretic framework in which entropy 
and energy flow are treated as fundamental degrees of freedom generating 
spacetime curvature \citep{Magnusson2025}. The theory aims to explain cosmic 
structure without invoking dark matter or dark energy, instead attributing 
gravitational effects to thermodynamic gradients.

A critical test for any alternative gravity theory is weak gravitational 
lensing, which probes the spacetime geometry directly through the deflection 
of light from distant galaxies. The Dark Energy Survey Year 6 (DES Y6) results 
\citep{DESY6_2026} report $S_8 = 0.789 \pm 0.012$, approximately $2.6\sigma$ 
lower than CMB-based predictions from Planck. This tension motivates testing 
whether modified gravity frameworks like EFC can provide a natural explanation.

However, as we demonstrate in this paper, EFC as currently formulated does 
\emph{not} automatically predict modified lensing. The entropy field $S(x)$ 
in EFC evolves according to its own dynamics without direct coupling to 
matter density perturbations. To obtain testable predictions, we must 
introduce an additional assumption.

The purpose of this paper is threefold:
\begin{enumerate}
    \item Demonstrate explicitly why EFC requires additional structure for lensing predictions
    \item Introduce a minimal, operationally motivated closure (Postulat A)
    \item Provide a complete falsification protocol against DES Y6 data
\end{enumerate}

We emphasize that this is a \emph{phenomenology} paper, not a claim that EFC 
explains observations. We provide the theoretical infrastructure for testing.

\section{EFC Field Equations}

We begin with the EFC action as presented in \citet{Magnusson2025}:
\begin{equation}
    S = \int d^4x \sqrt{-g} \left[ \frac{R}{16\pi G} 
    - \frac{\kappa_S}{2}(\nabla S)^2 - V(S) 
    - \frac{\kappa_J}{2}J_\mu J^\mu 
    + \gamma \nabla_\mu S \, J^\mu 
    + \lambda(\nabla_\mu J^\mu - \Sigma) \right]
\end{equation}
where:
\begin{itemize}
    \item $S(x) \in [0,1]$: normalized entropy potential
    \item $J^\mu$: energy/entropy flow four-vector
    \item $\Sigma \geq 0$: entropy production rate
    \item $\kappa_S, \kappa_J, \gamma$: dimensionless coupling constants
    \item $V(S) = V_0 + \frac{1}{2}m_S^2(S-S_0)^2$: entropic potential
\end{itemize}

Variation with respect to $g_{\mu\nu}$ yields the modified Einstein equations:
\begin{equation}
    G_{\mu\nu} = 8\pi G \left( T_{\mu\nu}^{(m)} + T_{\mu\nu}^{(S,J)} \right)
\end{equation}
with the entropy-flow stress-energy tensor:
\begin{align}
    T_{\mu\nu}^{(S,J)} &= \kappa_S \left[ \nabla_\mu S \nabla_\nu S 
    - \frac{1}{2}g_{\mu\nu}(\nabla S)^2 \right] - g_{\mu\nu} V(S) \nonumber \\
    &\quad + \kappa_J \left[ J_\mu J_\nu - \frac{1}{2}g_{\mu\nu} J_\alpha J^\alpha \right] 
    + \gamma \left[ \nabla_{(\mu} S \, J_{\nu)} - \frac{1}{2}g_{\mu\nu} \nabla_\alpha S \, J^\alpha \right]
\end{align}

The equations of motion for $S$ and $J^\mu$ are:
\begin{align}
    \kappa_S \Box S - V'(S) + \gamma \nabla_\mu J^\mu &= 0 \\
    \kappa_J J_\mu &= \gamma \nabla_\mu S + \nabla_\mu \lambda \\
    \nabla_\mu J^\mu &= \Sigma
\end{align}

\section{The Matter Coupling Problem}

\subsection{Why EFC Does Not Predict Lensing Automatically}

Combining equations (4) and (6), the entropy field evolves as:
\begin{equation}
    \kappa_S \Box S - V'(S) + \gamma \Sigma = 0
\end{equation}

\textbf{Critical observation}: The entropy production $\Sigma$ appears as the 
only source term for $\delta S$. However, in the EFC action as written, 
$\Sigma$ is \emph{not defined as a function of matter density}. It is a 
free function constrained only by $\Sigma \geq 0$.

This means that matter perturbations $\delta_m$ do not automatically source 
entropy perturbations $\delta S$. Without this coupling, the entropy sector 
evolves independently, and there is no mechanism by which structure formation 
modifies the gravitational potentials through entropy gradients.

\subsection{The Gap in the Current Framework}

To obtain a prediction for weak lensing, we need one of the following:
\begin{enumerate}
    \item[(A)] An explicit coupling $\Sigma = \Sigma(\rho_m, T, \ldots)$ 
    \item[(B)] A direct coupling in the action, e.g., $\int f(S) T \sqrt{-g}\, d^4x$
    \item[(C)] A coupling of $S$ to curvature scalars, e.g., $f(S)R$
\end{enumerate}

None of these appear in the published EFC field equations. This is not a 
criticism of the theory---it simply means the theory is incomplete for 
making lensing predictions in its current form.

\section{Postulat A: Minimal Operational Closure}

We introduce the following operational assumption:

\begin{quote}
\textbf{Postulat A}: Entropy production is linearly coupled to matter density:
\begin{equation}
    \delta\Sigma(k,z) = \xi \, \rho_m(z) \, \delta_m(k,z) \, g(z) \, h(k)
\end{equation}
where $\xi$ is a coupling constant, $g(z)$ is a regime-gating function, 
and $h(k)$ is a scale filter. Note that $\xi$ carries the dimensions 
required to make $\Sigma$ a scalar of dimension [time$^{-1}$], since 
$\kappa_S$, $\kappa_J$, and $\gamma$ are dimensionless and $S$ is normalized.
\end{quote}

This is explicitly an \emph{assumption}, not a derivation. We introduce it 
because:
\begin{itemize}
    \item It is the minimal structure needed to obtain predictions
    \item It is physically motivated (entropy production should correlate with matter)
    \item It yields a testable, falsifiable framework
    \item It can be replaced by a derived coupling in future work
\end{itemize}

For this paper, we adopt:
\begin{align}
    g(z) &= \Theta(z_t - z), \quad z_t = 2 \\
    h(k) &= \frac{1}{1 + k^2/k_*^2}, \quad k_* = 0.1 \, h/\text{Mpc}
\end{align}

The Heaviside function ensures no modification at $z > 2$, preserving early-universe 
physics (CMB, BBN). The scale filter concentrates effects on scales relevant to 
DES observations.

\section{Derivation of $\mu(k,z)$}

\subsection{Linear Perturbation Theory}

We perturb around a flat FRW background in Newtonian gauge:
\begin{equation}
    ds^2 = -(1+2\Psi)dt^2 + a^2(1-2\Phi)\delta_{ij}dx^i dx^j
\end{equation}

Perturbing $S = \bar{S}(t) + \delta S$ and using the quasi-static, 
sub-horizon approximation ($k \gg aH$, time derivatives subdominant):
\begin{equation}
    \left( \kappa_S \frac{k^2}{a^2} + m_S^2 \right) \delta S \approx \gamma \, \delta\Sigma
\end{equation}

Substituting Postulat A:
\begin{equation}
    \delta S \approx \frac{\gamma \xi \rho_m \delta_m \, g(z) \, h(k)}{\kappa_S k^2/a^2 + m_S^2}
\end{equation}

\subsection{Modified Poisson Equation}

In linear perturbation theory, the dominant contribution to $\delta\rho_S$ 
comes from the potential term:
\begin{equation}
    \delta\rho_S \approx V'(\bar{S}) \, \delta S = m_S^2 (\bar{S} - S_0) \, \delta S
\end{equation}

The modified Poisson equation becomes:
\begin{equation}
    k^2 \Psi = 4\pi G a^2 \left[ \rho_m \delta_m + \delta\rho_S \right] 
    = 4\pi G a^2 \, \mu(k,z) \, \rho_m \delta_m
\end{equation}

where we define:
\begin{equation}
    \boxed{\mu(k,z) = 1 + A_\mu \, \Theta(z_t - z) \, \frac{1}{1 + k^2/k_*^2}}
\end{equation}

with the amplitude:
\begin{equation}
    A_\mu = \frac{m_S^2 (\bar{S} - S_0) \, \gamma \xi}{\kappa_S k_*^2 + m_S^2}
\end{equation}

Equation (16) applies strictly in the linear perturbation regime. 
Non-linear corrections would require a full modified $N$-body treatment 
or an effective halo model extension.

\subsection{Gravitational Slip}

In this minimal framework, we find negligible anisotropic stress at linear order, 
yielding:
\begin{equation}
    \eta \equiv \frac{\Phi}{\Psi} \approx 1
\end{equation}

The lensing parameter is therefore:
\begin{equation}
    \Sigma_{\text{lens}}(k,z) = \frac{\mu(k,z)}{2}(1 + \eta) \approx \mu(k,z)
\end{equation}

\textbf{Important}: In this minimal model, the same function controls both 
growth and lensing. A richer model with gravitational slip would allow 
independent modification of these observables.

\section{DES Y6 Validation Protocol}

\subsection{Data}

We target the DES Y6 3$\times$2pt analysis \citep{DESY6_2026}:
\begin{itemize}
    \item Cosmic shear (shape-shape correlations)
    \item Galaxy clustering (position-position)
    \item Galaxy-galaxy lensing (position-shape)
\end{itemize}

Key measurements: $S_8 = 0.789 \pm 0.012$, $\Omega_m = 0.333$ (ΛCDM baseline).

\subsection{Model Comparison}

\textbf{Model 0} (Baseline): Standard ΛCDM with DES Y6 likelihood

\textbf{Model 1} (EFC-A): ΛCDM + $A_\mu$ with $z_t = 2$, $k_* = 0.1 \, h/$Mpc fixed

\subsection{Parameters and Priors}

\begin{center}
\begin{tabular}{lll}
\hline
Parameter & Prior & Notes \\
\hline
$\Omega_m, \Omega_b, h, n_s, A_s$ & DES Y6 standard & Cosmology \\
Shear calibration $m_i$ & DES Y6 standard & Nuisance \\
Photo-$z$ shifts & DES Y6 standard & Nuisance \\
Intrinsic alignments & DES Y6 standard & Nuisance \\
Galaxy bias $b_i$ & DES Y6 standard & Nuisance \\
$A_\mu$ & Flat $[-0.5, +0.5]$ & EFC parameter \\
\hline
\end{tabular}
\end{center}

\subsection{Pass/Fail Criteria}

\begin{center}
\begin{tabular}{lcc}
\hline
Criterion & Pass & Fail \\
\hline
$\Delta\chi^2$ (Model 1 vs Model 0) & $\geq 6$ & $< 2$ \\
$A_\mu$ significance & $> 2\sigma$ from 0 & consistent with 0 \\
AIC/BIC & prefers Model 1 & prefers Model 0 \\
Nuisance absorption & $A_\mu$ uncorrelated & $A_\mu$ degenerate with nuisance \\
\hline
\end{tabular}
\end{center}

\subsection{Control Tests}

To ensure robustness:
\begin{enumerate}
    \item Run cosmic shear only
    \item Run 2$\times$2pt (clustering + galaxy-galaxy lensing)
    \item Run full 3$\times$2pt
\end{enumerate}

If signal appears only in shear but vanishes in 3$\times$2pt, suspect systematics.

\subsection{CMB Consistency Check}

Compute CMB lensing power spectrum $C_\ell^{\phi\phi}$ with DES best-fit. 
If significantly discrepant with Planck CMB lensing, the model is ruled out.

\section{What Would Falsify This Framework}

We explicitly state the conditions under which this framework fails:

\begin{enumerate}
    \item \textbf{$A_\mu$ consistent with zero}: If DES Y6 data prefer $A_\mu = 0$, 
    then EFC with Postulat A provides no improvement over ΛCDM for weak lensing.
    
    \item \textbf{Wrong sign}: If data prefer $A_\mu > 0$ (stronger lensing), 
    this contradicts the hypothesis that EFC explains the low $S_8$ values.
    
    \item \textbf{Scale dependence mismatch}: If data prefer a $k$-dependence 
    inconsistent with equation (16), the specific form of Postulat A is falsified.
    
    \item \textbf{CMB lensing conflict}: If the best-fit model strongly 
    disagrees with Planck CMB lensing, the model is ruled out.
    
    \item \textbf{Nuisance degeneracy}: If $A_\mu$ is completely degenerate 
    with intrinsic alignments or photo-$z$ uncertainties, the test is inconclusive.
\end{enumerate}

\section{Limitations}

We are explicit about what this paper does \emph{not} do:

\begin{enumerate}
    \item \textbf{Postulat A is not derived}: The matter coupling is an operational 
    assumption. A complete theory would derive $\Sigma(\rho_m)$ from the action.
    
    \item \textbf{No data fit performed}: We present predictions and test design, 
    not results. The actual DES Y6 fit is future work.
    
    \item \textbf{No unique EFC signature}: Currently, $\mu(k,z)$ in equation (16) 
    resembles generic modified gravity parametrizations. Distinguishing EFC from 
    other MG theories requires additional observables or theoretical predictions.
    
    \item \textbf{Minimal slip}: We assume $\eta \approx 1$. A richer model with 
    $\eta \neq 1$ would allow independent probes of growth and lensing.
    
    \item \textbf{Fixed regime parameters}: We fix $z_t = 2$ and $k_* = 0.1 \, h/$Mpc 
    rather than fitting them, to avoid over-parametrization.
\end{enumerate}

\section{Discussion}

This paper occupies a specific position in the scientific literature: it is 
a \emph{phenomenology} paper that bridges formal theory and observational 
testing. We have:

\begin{itemize}
    \item Identified a gap in EFC (no automatic lensing prediction)
    \item Introduced a minimal closure (Postulat A)
    \item Derived testable predictions ($\mu(k,z)$)
    \item Designed a falsification protocol
\end{itemize}

This is analogous to how many modified gravity theories are developed: 
the action is proposed first, phenomenological consequences are worked out, 
and data comparison follows. We are at the second stage.

The value of this work is not in claiming EFC explains observations, but in 
providing the infrastructure to test whether it can.

\section{Conclusion}

We have presented a phenomenological framework for testing Energy-Flow Cosmology 
against weak lensing observations. The key results are:

\begin{enumerate}
    \item EFC as published does not automatically predict modified lensing
    \item Postulat A (linear entropy-matter coupling) provides minimal closure
    \item The resulting $\mu(k,z)$ is given by equation (16)
    \item A complete DES Y6 test protocol is provided with pass/fail criteria
\end{enumerate}

All code is available at \url{https://github.com/supertedai/EFC} and archived 
on Figshare.

The next step is to run the actual fit against DES Y6 data. Regardless of the 
outcome, this paper provides the theoretical foundation for that test.

\section*{Acknowledgments}

This work was developed in collaboration with the Symbiose AI research system.

\bibliographystyle{apalike}
\begin{thebibliography}{9}

\bibitem[DES Collaboration(2026)]{DESY6_2026}
DES Collaboration. (2026).
\newblock Dark Energy Survey Year 6 Results: Cosmological Constraints from 
Galaxy Clustering and Weak Lensing.
\newblock \emph{arXiv:2601.14559}.

\bibitem[Magnusson(2025)]{Magnusson2025}
Magnusson, M. (2025).
\newblock Energy-Flow Cosmology: Field Equations for Entropy-Driven Spacetime.
\newblock Figshare. \url{https://doi.org/10.6084/m9.figshare.30421807}

\bibitem[Jacobson(1995)]{Jacobson1995}
Jacobson, T. (1995).
\newblock Thermodynamics of Spacetime: The Einstein Equation of State.
\newblock \emph{Phys. Rev. Lett.}, 75, 1260.

\bibitem[Padmanabhan(2010)]{Padmanabhan2010}
Padmanabhan, T. (2010).
\newblock Thermodynamical Aspects of Gravity: New Insights.
\newblock \emph{Rept. Prog. Phys.}, 73, 046901.

\bibitem[Verlinde(2011)]{Verlinde2011}
Verlinde, E. (2011).
\newblock On the Origin of Gravity and the Laws of Newton.
\newblock \emph{JHEP}, 04, 029.

\end{thebibliography}

\end{document}
