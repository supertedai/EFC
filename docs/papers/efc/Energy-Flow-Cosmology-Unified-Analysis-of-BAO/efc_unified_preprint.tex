\documentclass[12pt,a4paper]{article}
\usepackage[utf8]{inputenc}
\usepackage{amsmath,amssymb,amsfonts}
\usepackage{graphicx}
\usepackage{booktabs}
\usepackage{hyperref}
\usepackage[margin=2.5cm]{geometry}
\usepackage{natbib}
\usepackage{needspace} % Forhindrer avsnitt fra å brytes over sider
\bibliographystyle{apalike}

% Forhindre enkeltlinjer øverst/nederst på sider
\widowpenalty=10000
\clubpenalty=10000
\raggedbottom

\title{Energy-Flow Cosmology: Unified Analysis of BAO, SN\,Ia, and RSD\\
with a Derived Effective Gravitational Coupling}

\author{Morten Magnusson\footnote{ORCID: \href{https://orcid.org/0009-0002-4860-5095}{0009-0002-4860-5095}}\\
\small Independent Researcher, 4051 Sola, Norway\\
\small \texttt{morten@magnusson.as}}

\date{January 2026\\[1em]
\small DOI: \href{https://doi.org/10.6084/m9.figshare.31215613}{10.6084/m9.figshare.31215613}}

\begin{document}

\maketitle

\begin{abstract}
We present a unified cosmological analysis testing Energy-Flow Cosmology (EFC) against 
combined baryon acoustic oscillation (BAO), Type Ia supernova (SN\,Ia), and redshift-space 
distortion (RSD) observations. The effective gravitational coupling $\mu(a) = G_{\rm eff}/G$ 
is \emph{derived} from the EFC field equation rather than phenomenologically fitted. 
We show that the entropy field $S(a)$ governing $\mu(a) = 1 + \beta S(a)$ produces a 
late-time enhancement in structure growth while leaving background expansion identical 
to $\Lambda$CDM (by construction, with $\gamma = 0$). With one free amplitude parameter 
($\beta = 0.16$) and transition hyperparameters fixed a priori, EFC achieves 
$\chi^2_{\rm total} = 51.1$ compared to $\Lambda$CDM's $\chi^2 = 49.4$ ($\Delta\chi^2 = +1.7$). 
The RSD fit is slightly worse but remains compatible within statistical fluctuations. 
This constitutes a first-pass regime-consistency check, demonstrating that EFC does not 
introduce internal tension between geometry and growth probes.
\end{abstract}

\vspace{1em}
\noindent\textbf{Categories:} Cosmology and Nongalactic Astrophysics (astro-ph.CO); 
General Relativity and Quantum Cosmology (gr-qc)

\vspace{0.5em}
\noindent\textbf{Keywords:} Energy-Flow Cosmology, modified gravity, effective gravitational 
coupling, baryon acoustic oscillations, redshift-space distortions, structure growth, 
entropy field, regime consistency

\section{Introduction}
\needspace{5\baselineskip}

The standard $\Lambda$CDM cosmological model successfully describes the expansion history 
of the Universe through baryon acoustic oscillations \citep{DESI2024} and Type Ia supernovae 
\citep{Brout2022}, as well as the growth of large-scale structure through redshift-space 
distortions \citep{Alam2021}. However, persistent tensions---notably the $S_8$ discrepancy 
between early and late-time probes \citep{DiValentino2021}---motivate exploration of 
alternatives.

Energy-Flow Cosmology (EFC) is a non-equilibrium thermodynamic framework in which a scalar 
energy-flow potential $E_f$ governs entropy-driven organization and spacetime curvature 
\citep{Magnusson2025a}. Unlike phenomenological modifications to gravity, EFC derives 
its effective gravitational coupling from first principles.

In this work, we perform the critical test: can the \emph{same} entropy field $S(a)$ 
that emerges from EFC theory simultaneously describe background expansion (BAO, SN) 
and structure growth (RSD)?

\section{Theoretical Framework}
\needspace{5\baselineskip}

\subsection{EFC Field Equation}

The EFC modification to general relativity takes the form:
\begin{equation}
G_{\mu\nu} = 8\pi G \left( T_{\mu\nu} + T^{(E_f)}_{\mu\nu} \right) + \Lambda_{\rm eff} g_{\mu\nu}
\end{equation}
where the energy-flow stress-energy tensor is:
\begin{equation}
T^{(E_f)}_{\mu\nu} = \alpha \left( \nabla_\mu E_f \nabla_\nu E_f - \frac{1}{2} g_{\mu\nu} (\nabla E_f)^2 \right)
\end{equation}

\subsection{Effective Gravitational Coupling}

From the EFC field equation, the modification to the Poisson equation for density 
perturbations yields an effective gravitational coupling:
\begin{equation}
G_{\rm eff}(S) = G \left( 1 + \beta S \right)
\end{equation}
where $\beta$ is the EFC coupling constant and $S$ is the normalized entropy field.

The key observable is:
\begin{equation}
\mu(a) \equiv \frac{G_{\rm eff}}{G} = 1 + \beta S(a)
\label{eq:mu}
\end{equation}

\subsection{Entropy Field Evolution}

The entropy field $S(a)$ evolves from $S \approx 0$ at early times (primordial equilibrium) 
to $S \to S_\infty$ at late times (structure saturation). A natural parameterization is:
\begin{equation}
S(a) = \frac{S_\infty}{2} \left[ 1 + \tanh\left( \frac{\ln a - \ln a_t}{\sigma} \right) \right]
\label{eq:entropy}
\end{equation}
where $a_t$ is the transition scale factor and $\sigma$ controls the transition width.

This sigmoid form is motivated by the thermodynamics of structure 
formation, where entropy production peaks during the L1$\to$L2 regime transition.

\subsection{Unified Prediction}

The crucial feature of EFC is that the \emph{same} entropy field $S(a)$ can determine both:
\begin{itemize}
\item Structure growth via $\mu(a) = 1 + \beta S(a)$
\item (Potentially) Background expansion via $\Lambda_{\rm eff}(a) = \Lambda_0(1 - \gamma S(a))$
\end{itemize}

For this analysis, we fix $\gamma = 0$ to isolate the growth modification, giving 
standard $\Lambda$CDM background expansion. This serves as a controlled test: can the 
same $S(a)$ modify growth without introducing internal tension with geometry probes?

\section{Methods}
\needspace{5\baselineskip}

\subsection{Observational Data}

We use three complementary probes:

\textbf{BAO}: DESI DR2 measurements of $D_M(z)/r_d$ at $z = 0.51, 0.71, 0.93, 1.32, 1.49, 2.33$ 
(6 data points).

\textbf{SN\,Ia}: Pantheon+ distance moduli at $z = 0.01$--$1.00$ (16 binned data points), 
with absolute magnitude $M$ analytically marginalized.

\textbf{RSD}: Compilation of $f\sigma_8(z)$ measurements from BOSS DR12, eBOSS DR16, and 
DESI Y1 at $z = 0.30$--$1.52$ (11 data points).

\subsection{Growth Equation}

The linear growth factor $D(a)$ satisfies:
\begin{equation}
D'' + \left(2 + \frac{H'}{H}\right) D' - \frac{3}{2} \Omega_m(a) \mu(a) D = 0
\end{equation}
where primes denote $d/d\ln a$. The growth rate is $f = d\ln D/d\ln a$, and the 
observable is $f\sigma_8(z) = f(z) \cdot \sigma_8 \cdot D(z)$.

\subsection{Model Comparison}

We compare:
\begin{enumerate}
\item \textbf{$\Lambda$CDM}: $\mu = 1$, standard $\Lambda$ (baseline)
\item \textbf{EFC}: $\mu(a) = 1 + \beta S(a)$, standard $\Lambda$ (growth modification only)
\end{enumerate}

The EFC parameters consist of one free amplitude and fixed transition hyperparameters:
\begin{itemize}
\item $\beta = 0.16$ (free amplitude, coupling constant)
\item $a_t = 0.55$ ($z_t = 0.82$, fixed a priori)
\item $\sigma = 0.10$ (fixed a priori)
\item $S_\infty = 1.0$ (normalization convention)
\end{itemize}

\section{Results}
\needspace{5\baselineskip}

\subsection{$\chi^2$ Analysis}
\needspace{10\baselineskip}

Table~\ref{tab:results} summarizes the fit quality for each probe.

\begin{table}[h!]
\centering
\caption{$\chi^2$ comparison between $\Lambda$CDM and EFC}
\label{tab:results}
\begin{tabular}{lcccc}
\toprule
Model & BAO $\chi^2$ & SN $\chi^2$ & RSD $\chi^2$ & Total $\chi^2$ \\
\midrule
$\Lambda$CDM & 9.81 & 28.98 & 10.56 & 49.35 \\
EFC ($\beta = 0.16$) & 9.81 & 28.98 & 12.28 & 51.06 \\
\midrule
$\Delta\chi^2$ & 0.00 & 0.00 & $+1.72$ & $+1.71$ \\
\bottomrule
\end{tabular}
\end{table}

\subsection{Key Findings}

\begin{figure}[p]
\centering
\includegraphics[width=\textwidth]{EFC_mu_theoretical_derivation.png}
\caption{Theoretical derivation of $\mu(a)$ from EFC field equations. 
\textbf{Top left:} Entropy field $S(a)$ evolution showing L0--L1 equilibrium, 
L1$\to$L2 transition, and L2--L3 structure formation regimes. 
\textbf{Top right:} Derived effective gravitational coupling $\mu(a) = 1 + \beta S(a)$ 
with $\beta = 0.16$. 
\textbf{Bottom left:} Resulting $f\sigma_8(z)$ prediction compared to RSD data. 
\textbf{Bottom right:} Derivation chain from EFC field equation to observable.}
\label{fig:derivation}
\end{figure}

\begin{figure}[p]
\centering
\includegraphics[width=\textwidth]{EFC_unified_analysis.png}
\caption{Unified cosmological analysis with locked $S(a)$ field. 
\textbf{Top row:} EFC entropy field, Hubble parameter $H(z)$, and BAO $D_M/r_d$. 
\textbf{Bottom row:} SN\,Ia distance moduli, RSD $f\sigma_8(z)$, and summary table. 
EFC with $\beta = 0.16$ achieves identical BAO/SN fits to $\Lambda$CDM while 
modifying structure growth through $\mu(a)$.}
\label{fig:unified}
\end{figure}

\textbf{Background probes (BAO, SN)}: Identical $\chi^2$ to $\Lambda$CDM by construction, 
since $\gamma = 0$ gives standard Friedmann expansion and $\mu(a)$ only modifies the 
Poisson equation for perturbations.

\textbf{Growth probe (RSD)}: $\Delta\chi^2 = +1.72$ compared to $\Lambda$CDM. EFC does 
not improve the RSD fit, but the deviation is small ($\Delta\chi^2 < 2$ for 11 data points) 
and remains compatible within statistical fluctuations.

\textbf{Regime consistency}: The same $S(a)$ field describes both geometry and growth 
without introducing internal tension.

\subsection{Physical Interpretation}

The EFC parameters have clear physical meaning:
\begin{itemize}
\item $\beta = 0.16$: 16\% enhancement in effective gravity at late times
\item $z_t = 0.82$: Transition epoch $\sim 7$ Gyr ago, during peak structure formation
\item $\sigma = 0.10$: Sharp L1$\to$L2 regime transition
\end{itemize}

The late-time enhancement in $G_{\rm eff}$ produces faster structure growth compared 
to $\Lambda$CDM. Whether this mechanism can address the $S_8$ tension requires 
dedicated weak lensing analysis beyond the scope of this work.

\section{Discussion}
\needspace{5\baselineskip}

\subsection{Theoretical Significance}

This analysis demonstrates that $\mu(a)$ is \emph{derived} from EFC field equations 
as a model specification, not phenomenologically fitted to data. The derivation chain:
\begin{equation}
\text{EFC field eq.} \to G_{\rm eff}(S) = G(1 + \beta S) \to \mu(a) = 1 + \beta S(a)
\end{equation}
The sigmoid form of $S(a)$ is motivated by thermodynamic transition phenomenology 
rather than derived from first principles.

\subsection{Comparison with Other Approaches}

Unlike generic $(w_0, w_a)$ parameterizations or Horndeski models with free functions, 
EFC has:
\begin{itemize}
\item One free amplitude parameter ($\beta$) with transition hyperparameters fixed a priori
\item A specified functional form for $\mu(a)$ motivated by thermodynamic principles
\item Unified treatment of perturbations through $S(a)$, with background optionally 
modified via $\gamma \neq 0$
\end{itemize}

\subsection{Limitations and Future Work}

The current analysis uses simplified likelihood evaluation. Future work should:
\begin{itemize}
\item Implement full MCMC analysis with covariance matrices
\item Include CMB constraints (Planck 2018)
\item Test weak lensing predictions ($S_8$)
\item Explore non-zero $\gamma$ for background modifications
\end{itemize}

\section{Conclusions}
\needspace{8\baselineskip}

We have performed a first-pass regime-consistency check of Energy-Flow Cosmology against 
combined BAO, SN\,Ia, and RSD observations. Our key findings:

\begin{enumerate}
\item The effective gravitational coupling $\mu(a) = 1 + \beta S(a)$ is \emph{derived} 
from EFC field equations, not phenomenologically fitted.

\item With one free amplitude ($\beta = 0.16$) and transition hyperparameters fixed 
a priori, EFC achieves $\chi^2_{\rm total} = 51.1$ compared to $\Lambda$CDM's 
$\chi^2 = 49.4$ ($\Delta\chi^2 = +1.7$).

\item Background probes (BAO, SN) are identical to $\Lambda$CDM by construction 
($\gamma = 0$).

\item The RSD fit is slightly worse ($\Delta\chi^2 = +1.7$) but remains compatible 
within statistical fluctuations.

\item The unified entropy field $S(a)$ describes both geometry and growth without 
introducing internal tension.
\end{enumerate}

This constitutes a first-pass consistency check. EFC does not improve upon $\Lambda$CDM 
in this analysis, but demonstrates that the framework can accommodate current observations 
without regime inconsistency. Full validation requires MCMC analysis with covariance 
matrices, CMB constraints, and dedicated weak lensing tests.

\section*{Data Availability}

Analysis code and data are available at \url{https://github.com/supertedai/EFC}.

\begin{thebibliography}{99}

\bibitem[Alam et al.(2021)]{Alam2021}
Alam, S., et al. 2021, Phys. Rev. D, 103, 083533

\bibitem[Brout et al.(2022)]{Brout2022}
Brout, D., et al. 2022, ApJ, 938, 110

\bibitem[DESI Collaboration(2024)]{DESI2024}
DESI Collaboration 2024, arXiv:2404.03002

\bibitem[Di Valentino et al.(2021)]{DiValentino2021}
Di Valentino, E., et al. 2021, Class. Quantum Grav., 38, 153001

\bibitem[Magnusson(2025a)]{Magnusson2025a}
Magnusson, M. 2025, Energy-Flow Cosmology v1.2, Figshare, doi:10.6084/m9.figshare.30563738

\end{thebibliography}

\end{document}
